% 纸张:A4
\hoffset=6.35mm
\hsize=146.5mm
\vsize=246.2mm

\input typezh
\baselineskip=13.5pt % 中文文档适合较大的行距

% 本文使用的额外的字体
\font\titlezh = "宋体" at 14.5pt
\font\titlerm = cmr7 scaled\magstep4
\font\tenkai = "楷体" at 10pt
\font\tenhei = "黑体" at 10pt
\font\tenss = cmss10
\font\eightrm = cmr8
\font\manual = manfnt scaled\magstep1

% 将中文字体设置为…
\def\kai{\let\zhfont\tenkai} % 楷体
\def\hei{\let\zhfont\tenhei} % 黑体

\def\title#1\par % 标题
{\vbox to .5in{\zhdualfont\titlezh\titlerm\vss\centerline{#1}\bigskip}}
\def\R{\raise .7ex\hbox % ®
{\ooalign{$\scriptstyle\bigcirc$\cr\fiverm\hss R\hss}}}
\def\hex#1{\hbox{\rm\H{}\tt#1}} % 十六进制格式
\def\<#1>{\langle\hbox{#1}\rangle} % 〈尖括号文字〉
\def\dbend{{\manual\char127}} % “危险弯道”
\def\danger{\noindent\hang\hangafter=-2 % “危险弯道”段落
\lower 1.5pt \hbox to 0pt{\hskip -\hangindent\dbend\hss}}
\def\@{\hskip\zhglue} % 手动修正被\@{分组\@}破坏的汉字间 glue
% FIXME 可否在 typezh 中解决这个问题?

\input rotate % 包括 \rotf
\def\XeTeX{\setbox0\hbox{E} % XeTeX logo
X\kern-.125em\lower .5ex\hbox{\rotf0}\kern-.1667em\TeX}
\def\sc{\eightrm}
\def\LaTeX{L\kern -.36em\raise .5ex\hbox{\sc A}\kern-.15em\TeX} % LaTeX logo

% |...| 原样排版 (多用于排版 TeX 代码)
\catcode`\|=\active
{\def\do#1{\catcode`\noexpand#1=12}
\xdef\uncatcodespecials{\dospecials}}
\outer\def|{\bgroup\uncatcodespecials\obeyspaces\tt\catcode`\|=2}

%%正文开始

\title typezh: \XeTeX\ 排版中文辅助宏

\XeTeX\ 是 \TeX\ 排版系统的扩展, 增加了对 Unicode 和 OpenType\R\ 的支持。
因此可被用于中文排版。

typezh 利用 |\XeTeXinterchartoks| 机制实现中文排版所需的字体自动切换%
和标点间距调整功能。 Plain \TeX\ 用户可以通过 $$\hbox{|\input typezh|}$$
加载这些功能。 \LaTeX\ 用户建议使用功能更加完善的 xeCJK 宏包或者 C\TeX\ 套装。

typezh 以能进行中文排版为目标, 提供尽可能少的功能, 而不是成为一个万能的
“宏包”。 和 Plain \TeX\ 一样, 仅提供了一个字号 (10 磅) 的字体。
用户如果需要更多的样式, 应当创建自己的格式文件, 并将 typezh 的代码
(可以做一些本地修改) 包含于其中。

\beginsection 1. 中西文字体切换

你可以自由地书写{%
\def\\#1{{\bfmt#1\efmt}}%
\def\bfmt#1{\tt\char`\{\string#1#1\space}%
\def\efmt{\/\tt\char`\}}%
\\{\rm 罗马体 Roman}、
\\{\it 意大利体 Italic}、
\\{\sl 倾斜体 Slanted}、
\\{\bf 粗体 Boldface} 和
\\{\tt 电报体 Teletype}。}

typezh 默认使用宋体排版所有中文, 这看起来比较单调, 字体渲染器给出的粗体%
和倾斜体的效果也不是很好。 另外, 在不同的系统上, 所需的字体名称可能并不是
“宋体”。 因此, 使用者应当修改源文件, 以符合自己的需求。
例如, 有些人可能希望使用\@{\kai 楷体\@}配合 {\it Italic\/},
{\hei 黑体\@}配合 {\bf Boldface}。\footnote{*}
{本文作者认为这是一种不恰当的设计, 因为\@{\hei 黑体\@}属于非衬线体
({\tenss sans-serif\/}), 不宜与衬线体 ({\bf serif\/}) 搭配。}

另外注意, 电报体属于等宽字体, 因此 typezh 进行了特殊处理 (将字体加宽 5\%),
使得一个汉字占据两个西文字符的宽度:
$$\vbox{\tt
% 下面的 \hbox 画了若干条竖线, 以体现中文和西文的对齐
\hbox{\dimen0=\ht\strutbox\advance\dimen0\baselineskip
\vrule height \dimen0 depth \dp\strutbox width 0pt
\dimen0=4em\advance\dimen0 by-.4pt
\def\do{\vrule\hskip\dimen0}%
\do\do\do\do\do\hskip-2.5em\do\do\do\unskip
}\vskip -2\baselineskip
%%
\hbox{The quick brown fox jumps over the lazy dog. The quick...}
\hbox{那只敏捷的棕毛狐狸跃过了那条懒狗。 那只敏捷的……}
}$$

\danger
当需要手动指定字体时, 可以使用 |\zhsinglefont|~宏禁用字体自动切换功能,
这样你就可以像正常使用 Plain \TeX\ 那样, 指定任意的字体。
例如, 上面的楷体文本可以使用 $$\hbox{|\zhsinglefont\tenkai 楷体|}$$ 得到。
(其中 |\tenkai| 是作者载入的 10 磅楷体字体。) 但是这样也将会使用该字体%
排版西文字符, 并很可能得到 {\tenkai some ugly letters.}
使用 $$\hbox{|\zhdualfont|}\<字体>$$ 可以仅设置中文 (实际上是 Unicode
字符集中的 “中日韩表意文字”) 及标点的字体, 而西文字体仍然使用当前的字体。
一个常见的用法是 $$\hbox{|\zhdualfont|}\<中文字体>\<西文字体>$$
同时设置中西文的字体; 这对于格式设计者而言是很有用的。
作者的建议是将这两个宏封装在格式中, 而不要出现在正文里。

\beginsection 2. 标点压缩

\begingroup\zhhalfwidth
由于中文字体中标点占据了一个字符的宽度,为了提高排版质量,需要对标点中的空白%
进行压缩。 typezh 默认将中文标点压缩至半个汉字的宽度(正如你所见,本段的标点%
就是这样的风格)。这和 \TeX\ 的西文标点采用了同样的约定:标点本身不含空白。%
作者建议用户在标点符号的旁边手动添加空格(尽管很多人并不习惯这样做),%
就和西文排版习惯一样。
\endgroup

\begingroup\zhfullwidth
另一些人使用另一种标点符号的风格。即不压缩标点,让它保持一个汉字的宽度%
(正如你所见,本段的标点就是这样的风格)。常在屏幕上阅读中文文本的用户%
可能会习惯于这种风格,但是从排版角度而言并不美观:“引号”和它左侧的冒号%
之间的空白看起来过大了。使用 |\zhfullwidth|~宏选择这种风格。%
要恢复默认的风格,使用 |\zhhalfwidth|。
\endgroup

一些用户可能习惯其他的风格, 例如: 标点符号默认为一个汉字的宽度,
但是位于行尾的标点将被压缩到半个汉字的宽度。 typezh 默认不支持这种风格,
用户需要修改源代码来指定她所需要的具体的规则。

\danger
当用户需要使用其他字号的字体时, 应当在载入字体之后使用 |\zhnormalskips|~宏,
重新设置标点压缩的大小。 格式设计者在定义诸如 |\tenpoint|、 |\ninepoint|
一类的宏时, 尤其要注意这一点。

\beginsection 3. 标点禁则

由于中文用户大多不在标点符号旁边添加空格, 除了带来空白压缩的问题以外,
还带来了禁则的问题——\TeX\ 不知道应该在哪里断行。 typezh 处理了这些情况,
使得 \TeX\ 可以在左标点的左侧或者右标点的右侧断行; 反过来的情况,
则禁止断行。

\beginsection 4. 数学公式中使用中文

在技术文档中, 偶尔会在公式中使用中文, 例如作为变量的下标。
typezh 允许直接在 \TeX\ 的数学模式中使用中文, 并以恰当的字体显示。
% 因为 |...| 不能作为 \displaylines 宏的参数,
% 所以用 box0 来暂存排版内容。
$$\setbox0\hbox{|F_浮 = \rho_液 g V_排|}
\displaylines{F_浮 = \rho_液 g V_排\cr \box0}$$

在目前的 typezh 实现中, 仅允许在公式中使用基本的 CJK 字符 (Unicode 范围
\hex{4E00}--\hex{9FFF})。 罕见的字符很少会出现在公式中。
另外, 不提倡在公式中使用中文标点, 因为逗号的水平位置和西文字母%
不太匹配, 而句号容易和小写字母~o 作下标弄混。

\beginsection 5. 字距调整

许多中文排版工具会在中西文之间自动添加空白, 以免中文和Western text之间%
过于拥挤, 影响美观。 typezh 默认不做这样的事, 简化了实现, 而且把控制权%
完全交给用户。 但是, 用户也可以选择设置 |\zhspace| 参数, 指定中西文之间%
自动插入的空白。

在每个汉字之间, typezh 会自动插入空白。 如果不这样做, 那么 \TeX\ 将无法%
通过调整字间距, 而使文本边界对齐。
(对于西文而言, 单词之间的空白是可以调整的。)
设置 |\zhglue| 参数调节汉字之间的间距。

\danger 但是, 上述的机制在较复杂的情况中经常失效, 尤其是遇到有分组或者 box
的情况。 例如, 设置 |\zhspace=.2em plus .2em minus .1em|,
则 $$\hbox{|使用\TeX 排版中文, 使用{\TeX}排版中文|}$$ 给出的结果是
$$\hbox{\zhspace=.2em plus .2em minus .1em
使用\TeX 排版中文, 使用{\TeX}排版中文}$$
这也是 typezh 默认不调整中西文间距的理由之一。
在一些罕见的情况下, 例如使用分组切换中文字体时, 应当在分组前后添加
|\hskip\zhglue|, 以使得该文本和周围文本的间距被正确处理。

\danger 例如, 在排版这篇文档时, 作者定义 |\def\@{\hskip\zhglue}|,
当作者在正文中插入\@{\bf 粗体字\@}, 而 \TeX\ 又不得不调整字间距时, 比较:
$$\vbox{\hbadness=10000
\setbox0\hbox{|\@{\bf 粗体字\@}|}
\setbox2\hbox{|{\bf 粗体字}|}
\halign{&\hss#\hss\cr
\box0\qquad&\hbox spread 3em{在正文中插入\@{\bf 粗体字\@}时……}\cr
\box2\qquad&\hbox spread 3em{在正文中插入{\bf 粗体字}时……}\cr
}}$$
分组还会破坏汉字之间的断行, 所以在汉字的上下文中使用分组时, 请务必小心。

\bye
