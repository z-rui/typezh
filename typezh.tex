%% XeTeX 排版中文辅助宏

\catcode`\@=11

% 常用中文字体
% Plain 定义的字体有:
% fam    0  1  2  3  4  5  6  7
%       rm  i sy ex it sl bf tt
% ten    x  x  x  x  x  x  x  x
% seven  x  x  x           x
% five   x  x  x           x
\font\tensong = "宋体" at 10pt % 略小于五号字 (10.5pt)
\font\sevensong = "宋体" at 7pt% 介于六号 (7.5pt) 和小六 (6.5pt) 之间
\font\fivesong = "宋体" at 5pt % 八号字 (5pt)
\font\tensongb = "宋体:embolden=1.4" at 10pt % 粗体
\font\tensongs = "宋体:slant=0.2" at 10pt    % 倾斜体
% 调整宽度,使得1个汉字=2个西文字符宽
\font\tensongx = "宋体:extend=1.05" at 10pt

\newfam\zhfam % 数学中的中文字体
\def\zhmathchars#1#2{\count@=#1\loop
  \Umathcode\count@=\z@ \zhfam \count@
  \ifnum\count@<#2 \advance\count@ by\@ne \repeat}

% 在数学环境中使用中文
% XXX 不含标点和扩展字符集
\zhmathchars{"4E00}{"9FFF}

% 启用中西文字体切换功能
\def\zhdualfont{\let\zhfont=}
% 禁用中西文字体切换功能(例如,希望使用中文字体中的西文字符时)
% 标点压缩和禁则功能不受影响
\def\zhsinglefont{\let\zhfont=\relax}

% 启用 XeTeX 字符间插入记号的功能,以便实现字体自动切换和标点压缩功能。
\XeTeXinterchartokenstate = 1

% 将引号设置为 CJK 标点
% 取消下列设置可以使用西文字体的引号(如果西文字体支持)。
\XeTeXcharclass`“=2 % 2: 左标点
\XeTeXcharclass`”=3 % 3: 右标点
\XeTeXcharclass`‘=2
\XeTeXcharclass`’=3

% 将一些全角符号设置为 CJK 字符
% XXX 尚无法处理它们的禁则
% TODO 根据需要补充更多符号
\XeTeXcharclass`·=1 % 1: 普通 CJK 字符
\XeTeXcharclass`—=1
\XeTeXcharclass`…=1
\XeTeXcharclass`¥=1

\def\zh@start{\begingroup
  \let\zh@start\relax
  \let\zh@end\endgroup
  \zhfont}

\newskip\zhspace% 中西文间距
\newskip\zhglue % 字间距
\newdimen\zhshrink % 标点压缩的量
\def\zhhalfwidth{\zhshrink=-.5em}% 半角标点
\def\zhfullwidth{\zhshrink=\z@}  % 全角标点

\def\zhnormalskips{\ifdim\zhshrink=\z@\else\zhhalfwidth\fi
  \zhspace=\z@ plus 0.1em
  \zhglue=\z@ plus 0.1em}

\def\zh@closepunct{\kern\zhshrink\allowbreak} % 右标点压缩右侧
\def\zh@openpunct{\null\kern\zhshrink}        % 左标点压缩左侧

\chardef\zh@boundary=4095 % 注意:旧版本的 XeTeX 为255
% 西文转中文
\XeTeXinterchartoks 0 1 {\zh@start\allowbreak\hskip\zhspace}
\XeTeXinterchartoks 0 2 {\zh@start\allowbreak\zh@openpunct}
\XeTeXinterchartoks 0 3 {\zh@start}
\XeTeXinterchartoks \zh@boundary 1 {\zh@start}
\XeTeXinterchartoks \zh@boundary 2 {\zh@start\zh@openpunct}
\XeTeXinterchartoks \zh@boundary 3 {\zh@start}
% 中文转西文
\XeTeXinterchartoks 1 0 {\zh@end\allowbreak\hskip\zhspace}
\XeTeXinterchartoks 2 0 {\zh@end}
\XeTeXinterchartoks 3 0 {\zh@closepunct\zh@end}
\XeTeXinterchartoks 1 \zh@boundary {\zh@end}
\XeTeXinterchartoks 2 \zh@boundary {\zh@end}
\XeTeXinterchartoks 3 \zh@boundary {\zh@closepunct\zh@end}
% 汉字之间的伸缩
\XeTeXinterchartoks 1 1 {\hskip\zhglue}
% 标点符号

% XXX \sfcode 仅对标点后确实紧随空格的情况生效。
% 建议使用“半角标点”+手动空格的方式调节标点符号的宽度。
% TODO 添加更多标点
\sfcode`。3000\sfcode`?3000\sfcode`!3000%
\sfcode`:2000\sfcode`;1500\sfcode`,1250%
\sfcode`、1100\sfcode`”0    \sfcode`’0%
\sfcode`)0

% 兼容模式:中文标点旁无空格时,处理禁则和压缩。
\XeTeXinterchartoks 1 2 {\allowbreak\hskip\zhglue\zh@openpunct}
\XeTeXinterchartoks 1 3 {}
\XeTeXinterchartoks 2 1 {}
\XeTeXinterchartoks 2 2 {\kern\zhshrink}
\XeTeXinterchartoks 2 3 {\nobreak\hskip\zhglue}
\XeTeXinterchartoks 3 1 {\zh@closepunct\hskip\zhglue}
\XeTeXinterchartoks 3 2 {\zh@closepunct\hskip\zhglue\zh@openpunct}
\XeTeXinterchartoks 3 3 {\kern\zhshrink}

% 重定义 Plain 的字体族,使其支持中文字体自动切换
\def\rm{\fam\z@\zhdualfont\tensong\tenrm} % 罗马字体对应宋体
\def\it{\fam\itfam\zhdualfont\tensongs\tenit} % 意大利体对应宋体斜体
\def\sl{\fam\slfam\zhdualfont\tensongs\tensl} % 倾斜体对应宋体斜体
\def\bf{\fam\bffam\zhdualfont\tensongb\tenbf} % 粗体对应宋体粗体
\def\tt{\fam\ttfam\zhdualfont\tensongx\tentt} % 打字机体对应宽(1.05倍)宋体

% 默认格式
\zhhalfwidth \zhnormalskips \rm
% 数学字体使用宋体
% 如需临时使用其他字体,请使用 \hbox{...}
\textfont\zhfam=\tensong
\scriptfont\zhfam=\sevensong
\scriptscriptfont\zhfam=\fivesong

% Plain TeX 默认每节首段不缩进,改成缩进
\outer\def\beginsection#1\par{\vskip\z@ plus.3\vsize\penalty-250
  \vskip\z@ plus-.3\vsize\bigskip\vskip\parskip
  \message{#1}\leftline{\bf#1}\nobreak\smallskip}

\catcode`@=12
